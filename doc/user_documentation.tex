\documentclass{article}[a4paper]
\usepackage{tcolorbox}
\usepackage[pdfborder={0 0 0}]{hyperref}

\makeatletter
\newcommand\subsubsubsection{\@startsection{paragraph}{4}{\z@}%
            {-2.5ex\@plus -1ex \@minus -.25ex}%
            {1.25ex \@plus .25ex}%
            {\normalfont\normalsize\bfseries}}
\makeatother
\setcounter{secnumdepth}{4}
\setcounter{tocdepth}{4}

\title{SWT21 lab kit \\ User manual}
\author{Binäs Teknik AB}

\begin{document}
\maketitle
\tableofcontents
\newpage

\begin{comment}

Använd ESP32 som HIL. Lägg till ett hårdvarukrestkort? LP-filter för analog-ut? (resistorstege?)
CAN-kommunikation är svårt, möjligen två transceivrar mellan Teensy och ESP32.
UART-kommunikation, LIN?
Två ESP32 för att testa mot varandra resp. en ESP32 som HIL och testa av den andra. Kan även köra micropython.

Med jumpers får vi med att ha koll på hårdvarukonfiguration.


Vda kan vi vilja testa?



All command and responses are '\\n'-terminated.
<module> <command> [arguments]
Max length

Common commands:
config set <parameter> <value> ...
config get <parameter>

Responses
---------
OK [data]
ERR <reason>

ERR Invalid command

ADC commands
------------
ADC<n> off
ADC<n> single
ADC<n> periodic <period (us)> [offset (us)]
ADC<n> status                      - OK ADC<n> status: <off/single/periodic <period> <offset>>
ADC<n> config set raw
ADC<n> config get raw              - OK ADC<n> raw: [on/off]
ADC<n> config set range <low> <high>
ADC<n> config get range            - OK ADC<n> range: <low> <high>
ADC<n> config timestamp <off/on>

Unsolicited ADC commands
------------------------
ADC<n> value [timestamp] <value>

DAC commands
------------
DAC<n> voltage
DAC<n> raw
DAC<n> config set range <low> <high>
DAC<n> config get range <low> <high>
pwm?

UART commands
-------------
UART<n> send <length>
UART<n>
UART<n> config set baudrate <baudrate>
UART<n> config get baudrate
UART<n> config set format <format>     - e.g. 8n1
UART<n> config get format

UART config parameters
----------------------
baudrate - <baudrate int32>
format - <format string> <bits><n|o|p><1|2>, e.g. 8n1 for 8 bits, no parity and 1 stop bit

Unsolicited UART commands
-------------------------
UART<n> overrun error <n>
UART<n> underrun error <n>
UART<n> framing error <n>
UART<n> parity error <n>

\end{comment}

\section{General}

The SWT21 lab kit allows communication and measurements using a single
microcontroller with a carrier board. The lab kit runs a firmware allowing a
computer to be connected as host which can then control the lab kit using the
host interface. This allows us to inspect network communication, interact with
hardware, etc. from the host computer.

The general connectivity of the lab kit is

\medskip
\renewcommand{\arraystretch}{1.5}
\begin{tabular}{|p{2cm}|p{1.5cm}|p{7cm}|}
\hline
test & test & Comment \\
\hline
WiFi & spec? & AP mode and STA mode \\
\hline
Bluetooth & 4.2? & \\
\hline
CAN & 2.0 & Only standard ID:s, 120 $\Omega$ termination via jumper configuration \\
\hline
LIN & 2.0 & Master or slave via jumper configuration \\
\hline
UART & & 3.3V maximum 2 Mbit/s, configurable 1-2 stop bits and parity bit \\
\hline
ADC & 2x & 0-3 V or 0-30 V via jumper configuration \\
\hline
DAC & 2x & 0-3 V or 0-12 V via jumper configuration (0-12 V) requires external power supply)\\
\hline
\end{tabular}


\subsection{Syntax rules}
The syntax used for commands in this manual is as follows

\begin{itemize}
\item <text> - indicates option to be replaced by other text
\item{[text] - optional argument to be replaced by other text if used}
\item{[text]... or <text>... - indicates multiple arguments may be used}
\item \{a|b\} - means either a or b may be used
\end{itemize}


\section{Host interface}
%configuration
%what is unsolicited commands
%syntax

The device is connected to a computer via USB. This connection provides a
USB-UART interface which presents itself as a serial port on the computer.

The communication parameters are:
\begin{itemize}
\item Baudrate: 2 Mbit/s
\item Stopbits: 1
\item Parity: None
\end{itemize}

\subsection{System requirements}

Windows driver, linux kernel, etc...

\url{https://www.silabs.com/developers/usb-to-uart-bridge-vcp-drivers}

\subsection{Protocol}

The communication interface uses text-based communication where each line is a
command or a response. Each line may be up to 255 bytes and ends with a new
line control character
(\textbackslash n). Other control characters will be ignored. The format of
the commands is as follows:
\begin{verbatim}
<module> <command> [arguments]...
e.g.: CAN config filter1 1f0 f70
\end{verbatim}

Responses can either be acknowledged with an
\begin{verbatim}
OK [data]
\end{verbatim}
, where data is any optional data returned by the command, or an error:
\begin{verbatim}
ERR <reason>
\end{verbatim}

General errors:
\begin{itemize}
\item ERR Invalid command
\item ERR Invalid arguments
\item ERR Command too long
\end{itemize}

Unsolicited commands are events sent from the lab kit to the host without a
preceding command asking directly for it. It is used for events like incomming
communication packets and periodic measuremnt data. The syntax is the same as
for commands but no response is expected.

\subsubsection{Boot}
On boot an informational string will be written on the host interface with
the following syntax

\begin{verbatim}
SWT21 lab kit
Booting...
Firmware version: <fw version>
Boot reason: <reason>
\end{verbatim}

Reason may be due to several reasons
\begin{itemize}
\item Power on
\item SW reset
\item WDT reset
\item Brownout
\item OS panic
\item Unknown
\end{itemize}

\section{Firmware update}
To update the firmware download and install the official flash download tool from
\url{https://www.espressif.com/en/support/download/other-tools}

Steps:
\begin{enumerate}
\item
\item
\end{enumerate}

\section{CAN}
The CAN bus can be used to send and recive 11-bit ID CAN frames.
When sending frames they can either be sent as single frames or as periodic
frames. Single frames are enqueued on the transmit buffer immediately after a
"CAN send" command has been sent. Periodic frames stores the frame data and
configuration in one
of eight periodic frame buffers. When enabled the periodic frame buffers will
send its payload periodically according to its configuration.


\subsection{CAN commands}
\subsubsection{CAN status}
\begin{tcolorbox}
	{\bf Syntax}

	\parshape 1 1cm \dimexpr\linewidth-2cm\relax
	CAN status

	\medskip
	{\bf Description}

	\parshape 1 1cm \dimexpr\linewidth-2cm\relax
	This command returns the current status of the CAN subsystem.

	\medskip
	{\bf Return values}

	\parshape 1 1cm \dimexpr\linewidth-2cm\relax
	OK CAN status: <rx on> <err frame count>
	\medskip \\
	{\it rx on} - the state of receiving CAN frames. \\
	{\it err frame count} - the total number of CAN errors received.
\end{tcolorbox}

\subsubsection{CAN rx}
\begin{tcolorbox}
	{\bf Syntax}

	\parshape 1 1cm \dimexpr\linewidth-2cm\relax
	CAN rx \{on|off\}

	\medskip
	{\bf Description}

	\parshape 1 1cm \dimexpr\linewidth-2cm\relax
	This command enables or disables receiving CAN frames

	\medskip
	{\bf Return values}

	\parshape 1 1cm \dimexpr\linewidth-2cm\relax
	OK
\end{tcolorbox}

\subsubsection{CAN send}
\begin{tcolorbox}
	{\bf Syntax}

	\parshape 1 1cm \dimexpr\linewidth-2cm\relax
	CAN send <can\_id>\#\{R|data\}

	\medskip
	{\bf Description}

	\parshape 1 1cm \dimexpr\linewidth-2cm\relax
	This command sends a single CAN frame on the CAN bus. On insufficient CAN
	memory it will return an overflow error.
	\medskip \\
	{\it can\_id} - the CAN ID in hexadecimal \\
	{\it R} - represents a remote frame \\
	{\it data} - is the frame data in hexadecimal

	\medskip
	{\bf Return values}

	\parshape 1 1cm \dimexpr\linewidth-2cm\relax
	OK \\
	ERR Invalid argument \\
	ERR Overflow
\end{tcolorbox}

\subsubsection{CAN config}
\begin{tcolorbox}
	{\bf Syntax}

	\parshape 1 1cm \dimexpr\linewidth-2cm\relax
	CAN config <config key> <arg>...

	\medskip
	{\bf Description}

	\parshape 1 1cm \dimexpr\linewidth-2cm\relax
	This command sets the configuration values of the CAN subsystem.
	Leaving out an argument returns the current value(s) with the same argument
	format.

	\medskip
	{\bf Return values}

	\parshape 1 1cm \dimexpr\linewidth-2cm\relax
	OK \\
	OK <value>... \\
	ERR Invalid argument
\end{tcolorbox}

\subsubsubsection{CAN config baudrate}
\begin{tcolorbox}
	{\bf Config key}

	\parshape 1 1cm \dimexpr\linewidth-2cm\relax
	baudrate

	\medskip
	{\bf Arguments}

	\parshape 1 1cm \dimexpr\linewidth-2cm\relax
	<baudrate> - CAN baudrate in bits/s (e.g. 125000)

	\medskip
	{\bf Description}

	\parshape 1 1cm \dimexpr\linewidth-2cm\relax
	This configuration sets the CAN baudrate.
\end{tcolorbox}

\subsubsubsection{CAN config filter<n>}
\begin{tcolorbox}
	{\bf Config key}

	\parshape 1 1cm \dimexpr\linewidth-2cm\relax
	filter0 \\
	filter1

	\medskip
	{\bf Arguments}

	\parshape 1 1cm \dimexpr\linewidth-2cm\relax
	<filter> <mask> \\
	<filter> - Filter value in hexadecimal (e.g. 3de)
	<mask> - Filter mask in hexadecimal (e.g. 7ff)

	\medskip
	{\bf Description}

	\parshape 1 1cm \dimexpr\linewidth-2cm\relax
	This configuration sets the values used by the acceptance filters.
\end{tcolorbox}

% TODO ########################### TODO ############### TODO ##############
periodic send with some storage buffers? add can tx {on|off} for periodic

\subsection{Unsolicited CAN commands}

\subsubsection{CAN frame}
\begin{tcolorbox}
	{\bf Syntax}

	\parshape 1 1cm \dimexpr\linewidth-2cm\relax
	CAN frame <can\_id>\#\{R|data\}

	\medskip
	{\bf Description}

	\parshape 1 1cm \dimexpr\linewidth-2cm\relax
	This command is sent when the system receives a CAN frame including its data
	\medskip \\
	{\it can\_id} - the CAN ID in hexadecimal \\
	{\it R} - represents a remote frame \\
	{\it data} - is the frame data in hexadecimal
\end{tcolorbox}

\subsubsection{CAN error frame}
\begin{tcolorbox}
	{\bf Syntax}

	\parshape 1 1cm \dimexpr\linewidth-2cm\relax
	CAN errframe <flag>

	\medskip
	{\bf Description}

	\parshape 1 1cm \dimexpr\linewidth-2cm\relax
	This command is sent when the system receives a CAN error frame
	\medskip \\
	{\it flag} - the error flag
\end{tcolorbox}


% ????????????????????? CAN rx\_overflow <n>

\begin{comment}

\section{LIN}
LIN commands
------------
LIN send <id>
LIN set txbuffer <n> <data>     - empty <data> disables TX buffer

Unsolicited commands:
LIN frame
error?

\end{comment}

\section{Command overview}
\begin{verbatim}
All command and responses are '\n'-terminated.
<module> <command> [arguments]...
Max length: 255 bytes

Common commands:
config <parameter> <value>... - set value(s)
config <parameter>            - get value(s)

Responses
---------
OK [data]
ERR <reason>

ERR Invalid command

CAN commands
------------
CAN status
    - OK CAN status: <rx on> <err frame count>

CAN rx {on|off}
CAN send <can_id>#{R|data}

CAN config parameters
---------------------
baudrate - (integer) baudrate
filter0 - (32-bit hex) filter, (32-bit hex) mask
filter1 - (32-bit hex) filter, (32-bit hex) mask

Unsolicited CAN commands
------------------------
CAN frame <can_id>#{R|data}
CAN errframe <flag>
\end{verbatim}

\begin{comment}
ADC commands
------------
ADC<n> off
ADC<n> single
ADC<n> periodic <period (us)> [offset (us)]
ADC<n> status
    - OK ADC<n> status: <off/single/periodic <period> <offset>>

ADC<n> config raw
ADC<n> config raw              - OK ADC<n> raw: [on/off]
ADC<n> config range <low> <high>
ADC<n> config range            - OK ADC<n> range: <low> <high>
ADC<n> config timestamp <off/on>

Unsolicited ADC commands
------------------------
ADC<n> value [timestamp] <value>

DAC commands
------------
DAC<n> voltage
DAC<n> raw
DAC<n> config range <low> <high>
DAC<n> config range <low> <high>
pwm?

UART commands
-------------
UART<n> send <length>
UART<n>
UART<n> config baudrate <baudrate>
UART<n> config baudrate
UART<n> config format <format>     - e.g. 8n1
UART<n> config format

UART config parameters
----------------------
baudrate - <baudrate int32>
format - <format string> <bits><n|o|p><1|2>
    e.g. 8n1 for 8 bits, no parity and 1 stop bit

Unsolicited UART commands
-------------------------
UART<n> overrun error <n>
UART<n> underrun error <n>
UART<n> framing error <n>
UART<n> parity error <n>
\end{comment}


\end{document}
